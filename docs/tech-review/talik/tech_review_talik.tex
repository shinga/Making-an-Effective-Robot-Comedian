\documentclass[onecolumn, draftclsnofoot,10pt, compsoc]{IEEEtran}
\usepackage{graphicx}
\usepackage{url}
\usepackage{setspace}

\usepackage{geometry}
\geometry{textheight=9.5in, textwidth=7in}

% 1. Fill in these details
\def \CapstoneTeamName{		AKA Robotics}
\def \CapstoneTeamNumber{		13}
\def \GroupMemberOne{}
\def \GroupMemberTwo{			Kevin Talik}
\def \GroupMemberThree{}
\def \CapstoneProjectName{		How to Make an Effective Robot Comedian}
\def \CapstoneSponsorCompany{	Oregon State University}
\def \CapstoneSponsorPerson{		Heather Knight}

% 2. Uncomment the appropriate line below so that the document type works
\def \DocType{		%Problem Statement
				%Requirements Document
				Technology Review
				%Design Document
				%Progress Report
				}
			
\newcommand{\NameSigPair}[1]{\par
\makebox[2.75in][r]{#1} \hfil 	\makebox[3.25in]{\makebox[2.25in]{\hrulefill} \hfill		\makebox[.75in]{\hrulefill}}
\par\vspace{-12pt} \textit{\tiny\noindent
\makebox[2.75in]{} \hfil		\makebox[3.25in]{\makebox[2.25in][r]{Signature} \hfill	\makebox[.75in][r]{Date}}}}
% 3. If the document is not to be signed, uncomment the RENEWcommand below
\renewcommand{\NameSigPair}[1]{#1}

%%%%%%%%%%%%%%%%%%%%%%%%%%%%%%%%%%%%%%%
\begin{document}

\bstctlcite{IEEEexample:BSTcontrol}
\begin{titlepage}
    \pagenumbering{gobble}
    \begin{singlespace}
        \hfill 
        % 4. If you have a logo, use this includegraphics command to put it on the coversheet.
        %\includegraphics[height=4cm]{CompanyLogo}   
        \par\vspace{.2in}
        \centering
        \scshape{
            \huge CS Capstone \DocType \par
            {\large\today}\par
            \vspace{.5in}
            \textbf{\Huge\CapstoneProjectName}\par
            \vfill
            {\large Prepared for}\par
            \Huge \CapstoneSponsorCompany\par
            \vspace{5pt}
            {\Large\NameSigPair{\CapstoneSponsorPerson}\par}
            {\large Prepared by }\par
            Group\CapstoneTeamNumber\par
            % 5. comment out the line below this one if you do not wish to name your team
            \CapstoneTeamName\par 
            \vspace{5pt}
            {\Large
                \NameSigPair{\GroupMemberOne}\par
                \NameSigPair{\GroupMemberTwo}\par
                \NameSigPair{\GroupMemberThree}\par
            }
            \vspace{20pt}
        }
        \begin{abstract}
          
          A comedian can observe an audience and improvise a delivery of a joke to connect the audience to the content. This makes the experience more authentic and genuine for the observer. The purpose of this project is to to discover what makes an entertaining interaction by studying a robot that performs comedy. We propose that a performance is enhanced when (1) the comedian interacts spontaneously with the audience, (2) the comedian has and conveys a coherent, well-developed character, and (3) the comedian adapts its act to cater to an audience based on their reaction. This document covers the technical requirements for our project, as well as a description of software, hardware, and outside limitations.

        \end{abstract}     
    \end{singlespace}
\end{titlepage}
\newpage
\pagenumbering{arabic}
\tableofcontents
% 7. uncomment this (if applicable). Consider adding a page break.
%\listoffigures
%\listoftables
\clearpage

% 8. now you write!
\section{Introduction}

  To make an effective robot comedian, we have developed three research questions that will be the basis of three internal systems for the machine. First, we are investigating how spontaneous interactions benefit a set. Second, we want to quantify the benefits of the robot's ability to personify it's character, and lastly how to adapt it's set corresponding to the audiences reaction. My role in this project is to develop an algorithm to adapt the robot's set of jokes based off of audience response.

  Timing and anticipation for jokes are crucial for the success of the set. Every joke that is told provides information to the audience about the robot, and from the opening joke, each new part of the set will build the repertoire of the bot. If a joke is well received by the audience, the relationship between the comedian and the crowd is strengthened, as the people will become more trusting of the content. Every joke will enable the audience to connote decisions, preferences and knowledge of the robot. This is where the connection, otherwise known as the "Theory of Mind", is made with the audience\cite{leslie}.

\section{Individual Role in Project}

Our group will be working collectively to make an effective robot comedian, but I will be specifically working on our third research question: the robots ability to adapt a set based off of audience response. From audience response to a joke, or bit, the algorithm should be able to determine the best fit for the next joke. The bit that tests the audience's preference for humor will be known as the "seed". The seed will be a small subset of jokes that represent the collective set of jokes. For example, the seed bit may have three jokes; one joke could be a self-depreciating joke, one could be a joke about food, and another could be a quick observational joke about the audience. If the audience responded well to the self-depreciating joke, the algorithm should choose the next joke should be at the expense of the robot. 

The seed of the set will give the audience pretense to the performance, and is the introduction of our robot. The delivery of the joke, and the content given has a large impact of robot character, and is out of the scope of the set creation; this is more suited toward the characterization research question. Also, the robot may need a more fluid way to interact with the audience (such as small talk, or crowd-work), which is underneath the breadth of the second research question about audience interaction. My contributions will be towards the system that determines which jokes, from our library of jokes, is best fit for our audience.

The algorithm will need to be able to input the strength of a delivered joke, and return a joke with attributes that match the strength of the joke. This will begin at the seed portion of the algorithm, and pick jokes until the set has lasted 3-6 minutes. The jokes will have some small variability in delivery, that correspond to the tasks of audience interaction and characterization.

\section{Technology overview}
    There have been a couple of previous studies of robot theatre, most notably Dr. Heather Knight \cite{KnightEightLessons:2011}, Katevas et al \cite{KatevasRobot:2014}, and Dr. Guy Hoffman \cite{hoffman2010anticipation}. To accomplish this task of designing an algorithm that can learn from the audiences' response to jokes, it is important to look at previous research, as well as the tools available for accomplishing this task.
\subsection{Previous Work}
  \subsection{ComedyParser}
  Katevas et al \cite{KatevasRobot:2014} has researched a robotic comedian agent previously with some success. During their research, they implemented a program called ComedyParser (https://github.com/minoskt/ComedyParser ) that collects audience response information from SHORE computer vision, and performs the stand up set. The decision components, or what the robot does with information gathered from the SHORE vision, will be most important for implementing an algorithm in our project. 
  A limitation with ComedyParser is that is specifically needs the SHORE vision to operate. SHORE will be too expensive for our project, and we will have to pursue more freely available systems. One solution that we had devised was to interprate the strength of the audience response to buttons on the robot. This will bypass the sensing component of comedy parser, as sensors for creating an audience model are out of the scope of this project.

  \subsection{Anticipation in Robot Theatre}
  Research conducted by Dr. Guy Hoffman studying the implications of anticipatory actions in social robotics \cite{hoffman2010anticipation}. This particular study found that humans working with a robot that can monitor anticipation for an event allows humans to anthropomorphize the robot with more human like attributes. This study uses non-atomic Markov Decision Processes (MDPs) to model the decisions for events. Additionally, Hoffman models anticipation with an impulse-cue situation, where the robot is waiting for an impulse to trigger a specific cue. This is non-deterministic, as the MDP process is modeled around the probability of an event happening.

  This will be helpful to use in our model, as Python, the main programming language for this project, has several implementations for MDPs, and the related automata "Context Free Grammar" (CFG).
\section{Tools}

  \subsection{Programming Languages}
  \subsection{Python}
  \subsection{java}

\section{Libraries}
  \subsection{Natural Language ToolKit}
  \subsection{PyKov: Markov Chains in Python}


\pagebreak


\bibliographystyle{IEEEtran}
\bibliography{refs}

\end{document}
