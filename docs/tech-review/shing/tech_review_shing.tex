\documentclass[onecolumn, draftclsnofoot,10pt, compsoc]{IEEEtran}
\usepackage{graphicx}
\usepackage{url}
\usepackage{setspace}

\usepackage{geometry}
\geometry{textheight=9.5in, textwidth=7in}

% 1. Fill in these details
\def \CapstoneTeamName{		AKARobotics}
\def \CapstoneTeamNumber{		13}
\def \GroupMemberOne{			Kevin Talik}
\def \GroupMemberTwo{			Anish Asrani}
\def \GroupMemberThree{			Arthur Shing}
\def \CapstoneProjectName{		How to Build an Effective Robot Comedian}
\def \CapstoneSponsorCompany{	Oregon State University}
\def \CapstoneSponsorPerson{		Heather Knight}

% 2. Uncomment the appropriate line below so that the document type works
\def \DocType{		Problem Statement
				%Requirements Document
				%Technology Review
				%Design Document
				%Progress Report
				}

\newcommand{\NameSigPair}[1]{\par
\makebox[2.75in][r]{#1} \hfil 	\makebox[3.25in]{\makebox[2.25in]{\hrulefill} \hfill		\makebox[.75in]{\hrulefill}}
\par\vspace{-12pt} \textit{\tiny\noindent
\makebox[2.75in]{} \hfil		\makebox[3.25in]{\makebox[2.25in][r]{Signature} \hfill	\makebox[.75in][r]{Date}}}}
% 3. If the document is not to be signed, uncomment the RENEWcommand below
%\renewcommand{\NameSigPair}[1]{#1}

%%%%%%%%%%%%%%%%%%%%%%%%%%%%%%%%%%%%%%%
\begin{document}
\begin{titlepage}
    \pagenumbering{gobble}
    \begin{singlespace}
    	% \includegraphics[height=4cm]{coe_v_spot1}
        \hfill
        % 4. If you have a logo, use this includegraphics command to put it on the coversheet.
        %\includegraphics[height=4cm]{CompanyLogo}
        \par\vspace{.2in}
        \centering
        \scshape{
            \huge CS Capstone \DocType \par
            {\large\today}\par
            \vspace{.5in}
            \textbf{\Huge\CapstoneProjectName}\par
            \vfill
            {\large Prepared for}\par
            \Huge \CapstoneSponsorCompany\par
            \vspace{5pt}
            {\Large\NameSigPair{\CapstoneSponsorPerson}\par}
            {\large Prepared by }\par
            Group\CapstoneTeamNumber\par
            % 5. comment out the line below this one if you do not wish to name your team
            \CapstoneTeamName\par
            \vspace{5pt}
            {\Large
                \NameSigPair{\GroupMemberOne}\par
                \NameSigPair{\GroupMemberTwo}\par
                \NameSigPair{\GroupMemberThree}\par
            }
            \vspace{20pt}
        }
        \begin{abstract}
        % 6. Fill in your abstract
				This project is about creating an effective robot comedian.
				The following is a brief statement of the project's problem, proposed solution, and performance metrics.
				The problem described is about the lack of a sense of liveliness in robotic interaction.
				In solution, the authors aim to create a robot comedian that reacts to audience reaction.
				The effectiveness of this will be measured after a series of studies on a live audience.
        \end{abstract}
    \end{singlespace}
\end{titlepage}
\newpage
\pagenumbering{arabic}
\tableofcontents
% 7. uncomment this (if applicable). Consider adding a page break.
%\listoffigures
%\listoftables
\clearpage

% 8. now you write!


\section{Lit Review (I will work on headings)}

Research on the improvement of HRI is indispensable for our project. In Heather Knight's \textit{Eight Lessons}, gestures, liveliness, and joke timing are all aspects that can be incorporated into the robot {\cite{KnightEightLessons:2011}}.
Relatable and appropriate gestures significantly helps improve communication between the robot and the audience. If the actions are predictable, humans can relate to the robot.
When watching someone perform an action, the human brain maps the actions onto itself and simulates the action in the best way possible. This is a physiological experience that should be replicated by the robot in order to enhance relatability. Simplicity is important as well. {\cite{KnightEightLessons:2011}}

In addition, Knight observed that having the robot portrayed as a living character rather than just an object that is kept up on stage improved the overall experience for the observers. Having believable interactions can enhance the feeling of a living character.
The goal of the audience tracking using sensors is to maximize enjoyment. The enjoyment levels were be read by the robot and used to modify upcoming jokes {\cite{KnightEightLessons:2011}}. Pausing and letting the audience laugh is vital as well. Starting the next joke too early can break the rhythm and leave the audience baffled. Looking around and body poses should be used to fill the pause {\cite{KnightEightLessons:2011}}.

In a previous study of robot comedy \cite{RobotComedyLab:2015}, Katevas found that when a robot engaged the audience through eye contact, the audience was more receptive to the performance. Eye contact from the robot is important, as it is a non-verbal cue for direct interaction. The audience members can identify that the machine is making an attempt to engage with specific members of the audience. We will investigate this further by having the robot perform various non-verbal and verbal interactions using sensors. The idea is to have the audience be a part of the performance even if they are not the ones performing. This can be accomplished if the robot is socially intelligent.

Other studies by Katevas et al. \cite{KatevasRobot:2014} that involved evaluating the social dynamics of a live performance by a robot have used SHORE\textsuperscript{TM} vision framework software to analyze and detect faces in the audience. SHORE\textsuperscript{TM} allows for facial expression recognition, estimated age, gender, and eye or mouth openings \cite{SHORE}, giving the study a heterogeneous audience model. These allowed for the robot to interact directly with specific audience members. However, usage of SHORE\textsuperscript{TM} involves expenses and funds that are unavailable to us, so we will encounter behavioral limitations dealing with a homogeneous audience model.

A study by Guy Hoffman {\cite{hoffman2010anticipation}} noted the importance of anticipation in human-robot interaction (HRI). The timing and meshing of anticipatory action and perception are a useful framework for HRI. The greatest challenge when designing a robot that will perform on stage is to enable the robot to be both - expressive and responsive. Robot models in the past have ended up on either extreme; they are either real-time and do not allow for continuous expression, or they are very animated but do not allow well times reactive behavior.


Researchers have also proposed multiple design patterns to promote sociability in Human-Robot Interaction (HRI). Some of these include having an initial introduction, some sort of didactic communication, including personal interests or history, and recovering from mistakes \cite{Kahn:2008}. These patterns in design are proposed to allow for more effective and meaningful social interactions. While there is yet to be much data or research on the validity of these claims, they may still prove to be useful in guiding the designs of our project.


% The effectiveness of our robot comedian will be evaluated by human enjoyment levels. Specifics regarding measurements and analysis will be later discussed with the client. Some possible methods include handing out surveys for the audience to fill out, which may include questions regarding the subjective reception of the robot comedian. Additionally, behavioral statistics may be used to evaluate the effectiveness of the comedy.

Robots utilizing non-verbal communication and statically written audience engagement have been attempted in robot comedy. In particular, Knight has observed the importance of character and spontaneous interactions in creating effective comedy \cite{KnightEightLessons:2011}. However, there is little research on the actual effectiveness of character and spontaneous interactions \cite{KatevasRobot:2014}. This project will aim to examine the effectiveness of character and spontaneous interactions in robot comedy.







\bibliographystyle{IEEEtran}
\bibliography{refs}

\end{document}
