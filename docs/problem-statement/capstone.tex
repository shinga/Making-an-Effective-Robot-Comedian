\documentclass[onecolumn, draftclsnofoot,10pt, compsoc]{IEEEtran}
\usepackage{graphicx}
\usepackage{url}
\usepackage{setspace}

\usepackage{geometry}
\geometry{textheight=9.5in, textwidth=7in,margin=0.75in}

% 1. Fill in these details
\def \CapstoneTeamName{		AKARobotics}
\def \CapstoneTeamNumber{		13}
\def \GroupMemberOne{			Anish Asrani}
\def \GroupMemberTwo{			Kevin Talik}
\def \GroupMemberThree{			Arthur Shing}
\def \CapstoneProjectName{		How to Make an Effective Robot Comedian}
\def \CapstoneSponsorCompany{	Oregon State University}
\def \CapstoneSponsorPerson{		Heather Knight}

% 2. Uncomment the appropriate line below so that the document type works
\def \DocType{		Problem Statement
				%Requirements Document
				%Technology Review
				%Design Document
				%Progress Report
				}

\newcommand{\NameSigPair}[1]{\par
\makebox[2.75in][r]{#1} \hfil 	\makebox[3.25in]{\makebox[2.25in]{\hrulefill} \hfill		\makebox[.75in]{\hrulefill}}
\par\vspace{-12pt} \textit{\tiny\noindent
\makebox[2.75in]{} \hfil		\makebox[3.25in]{\makebox[2.25in][r]{Signature} \hfill	\makebox[.75in][r]{Date}}}}
% 3. If the document is not to be signed, uncomment the RENEWcommand below
%\renewcommand{\NameSigPair}[1]{#1}

%%%%%%%%%%%%%%%%%%%%%%%%%%%%%%%%%%%%%%%
\begin{document}
\begin{titlepage}
    \pagenumbering{gobble}
    \begin{singlespace}
        \hfill
        % 4. If you have a logo, use this includegraphics command to put it on the coversheet.
        %\includegraphics[height=4cm]{CompanyLogo}
        \par\vspace{.2in}
        \centering
        \scshape{
             \huge CS Capstone \DocType \par
            {\large\today}\par
            \vspace{.5in}
            \textbf{\Huge\CapstoneProjectName}\par
            \vfill
            {\large Prepared for}\par
            \Huge \CapstoneSponsorCompany\par
            \vspace{5pt}
            {\Large\NameSigPair{\CapstoneSponsorPerson}\par}
            {\large Prepared by }\par
            Group\CapstoneTeamNumber\par
            % 5. comment out the line below this one if you do not wish to name your team
            \CapstoneTeamName\par
            \vspace{5pt}
            {\Large
                \NameSigPair{\GroupMemberOne}\par
                \NameSigPair{\GroupMemberTwo}\par
                \NameSigPair{\GroupMemberThree}\par
            }
            \vspace{20pt}
        }
        \begin{abstract}
        % 6. Fill in your abstract




        The purpose of this project is to make a robot that performs comedy. Robot interaction lacks the traits of human interaction. Character, authenticity, and liveliness are some traits of human interaction that enables people to stay invested in an interaction, and are important aspects of successful stand-up comedy. In an effort to create an effective robot comedian, these traits should be present in the robot. The challenge will be to make the robot feel and perform more like a human would while keeping the audience engaged. This can be overcome by considering audience feedback after every joke or story that is told and using artificial intelligence to adjust future jokes. We propose that the robot should be able to construct its own jokes according to these traits, and also in accordance with audience reaction. The robot’s performance will be measured using audience sensing tools.

        \end{abstract}
    \end{singlespace}
\end{titlepage}
\newpage
\pagenumbering{arabic}
\tableofcontents
% 7. uncomment this (if applicable). Consider adding a page break.
%\listoffigures
%\listoftables
\clearpage

\section{The Problem}

“The jostling of ideas... produces a physical jostling of our internal organs and we enjoy that physical stimulation.”  {Morreall}

Humor is an entertaining break in expectations and can happen in any interaction. The element of incongruity has long been acknowledged to be an essential part of humor, and has been noted by philosophers such as Aristotle and Kant {\cite{StanfordHum:2016}}. A form of modern comedy that capitalizes on this element of incongruity is stand-up comedy.

In stand-up comedy, comedians have to rely on their scripted jokes and their ability to improvise to have a successful performance. In many ways, a robot interaction can be compared to a stand-up set. Within the context of stand up comedy, a robot must be receptive to an audience and tell jokes to make an audience laugh. Some robots, such as an ATM or an automated cashier, require successful interactions to complete tasks. These machines are not funny, and the interaction is less significant.

However, even if a task is unsuccessful, the attitude towards a robot can be positively connotated by a more responsive and empathetic machine. {\cite{DesignExBeh:2017}}

There have been studies on the comedic value of jokes told by a robot. In one study, Sjöbergh and Araki {\cite{RobotsMakeThings:2008}} examined the significance of having a robot tell a joke. However, this study evaluated joke performance by a robot, but not an entire stand-up set. In addition, another study by Katevas et al. {\cite{RobotComedyLab:2015}} evaluated the influence of non-verbal aspects of joke delivery. To extend on these lines of research, we intend to focus on what character qualities robot express during a stand-up performance, both verbally and physically to an audience.

While robots can perform comedy to some extent, there is still a significant gap between human and robot performances. One of the biggest factors leading to this is that robots have limitations detecting and reacting to the dynamics of the audience behavior {\cite{KatevasRobot:2014}}. This may go unnoticed in shorter sets but when performing for extended periods, this would be very prominent and could potentially leave the audience uninterested.

Existing research with robot comedians has given rise to the questions of the nature of comedy: What does robot comedy need to succeed? What is it that makes a performance engaging and alive? How might comedy exclusive to a robotic comedian improve its act {\cite{RobotsMakeThings:2008}}? To answer these questions, we hypothesize that a performance is enhanced when (1) the comedian interacts spontaneously with the audience, (2) the comedian has and conveys a coherent, well-developed character, and (3) the comedian adapts its act to cater to an audience based on their reaction.




%We are trying to make a robot that performs stand-up comedy and entertain people. This should be done while making the robot as close to human as possible. The comedy routine will be scripted but there should be some sort of interactions between jokes to make the performance feel more real. The audience should feel engaged throughout the performance.



\section{Proposed Solution}

To make robot comedy successful, we hypothesize that the comedian-audience interaction feel authentic and engaging is the perceived character through improvisational dialogue. We intend to evaluate an audience's experience during a robot comedy set to interpret how a robot’s character is received.

In a final performance, we want to see an audience who is responsive to the robot comedian, and a robot that is adaptively communicating a set of jokes to a dynamic audience. To build up to this, we want to see separately, a robot that can functionally tell jokes through a personality, and then view how an live audience responds to the delivery. In order to make this feasible, some steps will first be taken locally and remotely.

Locally in our group, we want to determine what a robot can be technically limited to, and how to structure the software behind character parameters and verbal sentences. Through this process, we can design jokes and set segments. We will also design character personas that the robot will use, and test sets ourselves with our friends and peers to ensure the hilarity of the set.

Outside of our groups, we want to see how our jokes are communicating to the audience. Not only do we want to see jokes performed by the robot, but we also want to test our jokes performed by a human, and compare the response. Once we can see how an audience interacts with a robot, we want to observe decisions that can the robot can make to engage the audience and adapt to measured responses.

In the “Real World”, our robot performance will make sets dynamically, and observe audiences to choose jokes adaptively; the robot will be socially intelligent. Additionally, in line with work done by Katevas et al. {\cite{RobotComedyLab:2015}} and Sjöbergh and Araki {\cite{RobotsMakeThings:2008}}, the robot should make use of non-verbal cues. This includes looking at the audience while talking to them, making appropriate hand gestures to supplement the jokes, and walking back and forth on stage. These aspects of body language will help keep the audience engaged and give the robot a human touch.


\section{Performance Metrics}

Broad metrics involved will include the audience’s evaluation of the robot’s character and response to jokes. We also need to measure the robot’s ability to deliver jokes and adapt to a response.

The robot will be tested by giving it different personas at different performances. To see if the robot’s character was conveyed coherently, the audience will fill out a questionnaire prompting them to describe its character, as well as some humanizing questions, e.g. “Would you invite this robot to dinner?” These responses will also be used to study if the robot matched the expected persona.

Audience response to jokes will also be measured by behavioral observation. With consent, the audience may be recorded for their reactions, such that responses can be retrospectively observed to see which parts of the set worked and which parts were less successful.

\bibliographystyle{./IEEEtran}

\bibliography{refs}
\end{document}
