\documentclass[onecolumn, draftclsnofoot,10pt, compsoc]{IEEEtran}
\usepackage{graphicx}
\usepackage{url}
\usepackage{setspace}

\usepackage{tikz}
\usetikzlibrary{%
	arrows, backgrounds, calc,%
    patterns, positioning, shapes.geometric%
}
\RequirePackage{pgfcalendar}
\usepackage{pgfgantt}

\usepackage{geometry}
\geometry{textheight=9.5in, textwidth=7in,margin=0.75in}
\bibliographystyle{IEEEtran}

% 1. Fill in these details
\def \CapstoneTeamName{			Laughtimus Prime}
\def \CapstoneTeamNumber{		13}
\def \GroupMemberOne{			Anish Asrani}
\def \GroupMemberTwo{			Kevin Talik}
\def \GroupMemberThree{			Arthur Shing}
\def \CapstoneProjectName{		How to Make an Effective Robot Comedian}
\def \CapstoneSponsorCompany{	Oregon State University}
\def \CapstoneSponsorPerson{		Heather Knight}

% 2. Uncomment the appropriate line below so that the document type works
\def \DocType{
% Problem Statement
				Requirements Document
				%Technology Review
				%Design Document
				%Progress Report
				}

\newcommand{\NameSigPair}[1]{\par
\makebox[2.75in][r]{#1} \hfil 	\makebox[3.25in]{\makebox[2.25in]{\hrulefill} \hfill		\makebox[.75in]{\hrulefill}}
\par\vspace{-12pt} \textit{\tiny\noindent
\makebox[2.75in]{} \hfil		\makebox[3.25in]{\makebox[2.25in][r]{Signature} \hfill	\makebox[.75in][r]{Date}}}}
% 3. If the document is not to be signed, uncomment the RENEWcommand below
%\renewcommand{\NameSigPair}[1]{#1}

%%%%%%%%%%%%%%%%%%%%%%%%%%%%%%%%%%%%%%%

\begin{document}

\bstctlcite{IEEEexample:BSTcontrol}

\begin{titlepage}
    \pagenumbering{gobble}
    \begin{singlespace}
        \hfill
        % 4. If you have a logo, use this include graphics command to put it on the coversheet.
        %\includegraphics[height=4cm]{CompanyLogo}
        \par\vspace{.2in}
        \centering
        \scshape{
             \huge CS Capstone \DocType \par
            {\large\today}\par
            \vspace{.5in}
            \textbf{\Huge\CapstoneProjectName}\par
            \vfill
            {\large Prepared for}\par
            \Huge \CapstoneSponsorCompany\par
            \vspace{5pt}
            {\Large\NameSigPair{\CapstoneSponsorPerson}\par}
            {\large Prepared by }\par
            Group\CapstoneTeamNumber\par
            % 5. comment out the line below this one if you do not wish to name your team
            \CapstoneTeamName\par
            \vspace{5pt}
            {\Large
                \NameSigPair{\GroupMemberOne}\par
                \NameSigPair{\GroupMemberTwo}\par
                \NameSigPair{\GroupMemberThree}\par
            }
            \vspace{20pt}
        }
        \begin{abstract}
        % 6. Fill in your abstract




A comedian can observe an audience and improvise a delivery of a joke to connect the audience to the content. This makes the experience more authentic and genuine for the observer. The purpose of this project is to to discover what makes an entertaining interaction by studying a robot that performs comedy. We propose that a performance is enhanced when (1) the comedian interacts spontaneously with the audience, (2) the comedian has and conveys a coherent, well-developed character, and (3) the comedian adapts its act to cater to an audience based on their reaction. This document covers the technical requirements for our project, as well as a description of software, hardware, and outside limitations.


        \end{abstract}
    \end{singlespace}
\end{titlepage}
\newpage
\pagenumbering{arabic}
\tableofcontents
% 7. uncomment this (if applicable). Consider adding a page break.
%\listoffigures
%\listoftables
\clearpage

\section{Introduction}

\begin{ganttchart}{1}{12}
\gantttitle{2011}{12} \\
\gantttitlelist{1,...,12}{1} \\
\ganttgroup{Group 1}{1}{7} \\
\ganttbar{Task 1}{1}{2} \\
\ganttlinkedbar{Task 2}{3}{7} \ganttnewline
\ganttmilestone{Milestone}{7} \ganttnewline
\ganttbar{Final Task}{8}{12}
\ganttlink{elem2}{elem3}
\ganttlink{elem3}{elem4}
\end{ganttchart}

\subsection{Background}

Nothing to see here.

\subsection{Introduction}
In previous studies, robots utilizing non-verbal communication, as well as attempts at adaptive robots comedians have been done. In particular, Heather Knight has observed the importance of character and spontaneous interactions in creating effective comedy. However, there is little research on the actual effectiveness of character and spontaneous interactions. This project will aim to examine the effectiveness of character and spontaneous interactions in robot comedy.

We will try to answer the effectiveness of the following questions:
\begin{enumerate}[\IEEEsetlabelwidth{6)}]
\item How can the robot integrate the audience to make them feel like a part of the performance?
\item How can the robot convey and a coherent and well-developed character?
\item How can the robot adapt and influence to the audience?
\end{enumerate}


\subsection{"Plan of attack" (We need to think of a better term)}
This project will be carried out in three phases. Ideally, these phases will correspond to the three terms allocated for this class; A learning and exploration phase in Fall term, a prototyping and testing phase in the Winter, and our evaluation phase Spring term.

\subsubsection{Learning/Exploration Phase}
In the learning and exploration phase, most of what will be done is to prepare us for the prototyping and testing phase. The three of us will become familiar with the coding environment by self learning and also through our meetings with Heather, the client. We will test primitive scripts and code on the NAO robot, to learn how the coding environment works and to familiarize ourselves with hardware limitations. Additionally, we will learn to work with the sensors on the robot, and how they function. We will also research mechanisms behind jokes and comedy, and become familiar with research and studies on comedic interactions. We will develop preliminary research questions as subjects of interest for our study. These steps will ideally be completed by the end of Fall term, in early December. As a stretch goal, we also hope to work with machine learning and adaptive behaviors that can be implemented into the NAO robot.

\subsubsection{Prototyping/Testing Phase}
In the prototyping and testing phase, we will develop early prototypes based on the preliminary research questions created in the previous phase. As of the time of writing, these questions include the effectiveness of implementing crowdwork and the effectiveness of implementing a discernable character. Crowd-work will involve simple audience sensing, as well as jokes that incorporate a measurement of response from the audience. Character implementation will involve testing the differences in effectiveness of robot vs human joke delivery, and the effectiveness of robo-centric jokes.

As a stretch goal, we also hope to prototype and test the effectiveness of adapting a set to the audience, using intelligent calibration of the sensors. These prototypes will be in the form of set scripts, and will be tested in front of a small sample of humans, or in the form of a video recording. Feedback from our testing will influence the direction of our prototyping, meaning that the implementation of our research questions may vary according to the response.

\subsubsection{Evaluation Phase}
While doing the research, we will perform 9 shows with audiences ranging from 30-50 people. These shows would be at an organized location with an audience or potentially in the form of a street performance.

\subsection{Methods}
We will analyze the audience enjoyment levels using sensors in the robot. We will also take note of what aspects of the robot got the most laughs of the audience and how it was effective. Multiple different robot personalities will be tested to see what kind of character appeals to the audience. The audience will also be surveyed to determine what aspects of the robot were enjoyable and what did not appeal to the audience.

The effectiveness of our robot comedian will be evaluated by human enjoyment levels. To do this… Idk the specifics. Maybe a survey, auditory sensors, etc.

Something about statistics. We will run statistics on our accumulated data. Or something.


\subsection{Potential Discussions}

Nothing to see here.

\subsection{Timeline}

Insert Gantt chart.


% \bibliographystyle{IEEEtran}
% \bibliography{refs}
\end{document}
